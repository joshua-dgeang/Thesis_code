\documentclass[12pt]{article}
\usepackage{amsmath}
\usepackage{amssymb}
%\usepackage[margin=1.2 in, includefoot]{geometry}
\usepackage{color}
\usepackage{rotating}
%\usepackage{setspace}
\usepackage{graphicx}
\author{Zimian Zhang}
\title{Chapter 4: A Two-Asset Cash Management Model}
\begin{document}

%\begin{spacing}{1.0}
\maketitle
\setcounter{page}{0}
\thispagestyle{empty}


\newpage
\tableofcontents
\setcounter{page}{0}
\thispagestyle{empty}

\newpage

\section{Introduction}
\section{Problem Description and Modelling Approach}
\subsection{Problem description}
\subsubsection{Uncontrolled process}
Let $x_t \in X$ and $y_t \in Y$ denote the cash level and the asset level of the company at time $t \in T$ where T is the infinite time horizon. Let the dynamics of uncontrolled cash-level consists of two parts: an exogenous process $x_{ex}(t)$ and an endogenous variable $x_{en}(t)$. We use $x_{ex}(t)$ to describe the environmental factors affecting the cash flows and use $x_{en}(t)$ to describe the cash inflows generated by companies' profitable asset. 



A reasonable assumption is that the exogenous process $x_{ex}(t)$ can be approximated as a Wiener process with a constant drift ($\mu$) and a constant variance ($\sigma$), i.e. $$ \left\{ 
\begin{aligned}
&x_{ex}(t) = \mu dt + \sigma d W_t \\
&x_{ex}(0) = x_0\\
\end{aligned}
\right.$$
In the above equation, $W_t$ is a standard one-dimensional Wiener process.
We also assume that the endogenous variable $x_{en}(t)$ is proportional to the size of companies' asset account, i.e. $$ \left\{ 
\begin{aligned}
&\frac{dx_{en}(t)}{dt} = ry\\
&x_{en}(0) = 0\\
\end{aligned}
\right.$$ where $r$ is the business return rate of profitable asset. In reality, cash might also generate some profit, for instance, the current accounts' interest. But it is trivial in the comparison of the profits from assets. The uncontrolled cash state process is the superposition of both exogenous and endogenous process, i.e. $x(t) = x_{ex}(t) + x_{en}(t)$. On the other hand, we assume that the uncontrolled assets account is constant over the time horizon.


\subsubsection{Actions and policy}
At any point of the time, the manager can control his cash-holding level by transferring money between two accounts with some transaction fees. Let $A$ be the set of all possible actions the manager could take over the time horizon. Denote the feasible actions at each decision epoch by $A_{x_t, y_t}: A_{x_t, y_t} \subseteq A$. Let $f_n(t)$ be the decision function at $n^{th}$ epoch at time $t$, which could be stationary, i.e. $f_n(t): X \times Y \mapsto  A_{x_t, y_t}$ or history dependent, i.e. $f_n(t): H_t \mapsto  A_{x_t, y_t}$ where $H_t$ is the set of history. A policy $\pi$ is a sequence of decision functions used at each decision epoch, i.e. $\pi = \{f_1(t_1), f_2(t_2),... \}$. {\color{red}(Can we prove that in this model a stationary policy is as good as a historian policy?)}

\subsubsection{Cash management cost}
There are two sources of cash management cost in our model: the cost generated by cash transfers between two accounts (i.e. transaction cost) and the cost associated with cash shortages.
Let the transaction cost function be partially fixed and partially proportional to the transfer amount (i.e. $a_t:a_t \in A_{x_t, y_t}$). Let a positive transfer amount be moving money from asset account to cash account and a negative amount the other way around. The cost generated by the $n^{th}$ action which was taken at time $t$ can be write as: 
$$ \Gamma_n(a_t)=\left\{
\begin{array}{rcl}
 K_1 + k_1 a_t       &      & {a_t < 0}\\
0     &      & {a_t = 0}\\
 K_2 + k_2 a_t       &      & {a_t > 0}
\end{array} \right. .$$ We assume that once the cash-level $x_t$ is negative, the manager will be forced to sell his asset to offset the cash deficit and to pay a fixed amount of penalty ($P$). Now let the $SC_m(x_t)$ be the cost caused by $m^{th}$ time of cash shortage which happened at time $t$ and it is given by: $$SC_m(x_t)= 1_{\{x_t < 0\}} \cdot \left\{ \Gamma(|x_t|) + P\right\}.$$ Denote the total number of actions and cash shortages over the whole time horizon by $N$ and $M$ respetively ($N \leq \infty, M \leq \infty$). Then the total cash management cost following policy $\pi$ over the whole time horizon can be written as: $$ TC^{\pi} = \sum^N_{n=1} \Gamma_n  +  \sum^M_{m=1} SC_m, \qquad N\leq\infty,M\leq\infty.$$

\subsubsection{Objective function}
Assume the manager want to maximise the net profits of his business over the whole time horizon. Given initial cash level $x$ and asset level $y$, the undiscounted total expected utility value following policy $\pi$ is given by:

$$V^\pi(x,y) = \mathbb{E}^\pi\left[ \left. \int_0^ \infty x(t)   dt - \sum^N_{n=1} \Gamma_n  -  \sum^M_{m=1} SC_m  \right| x(0) = x, y(0) = y\right].$$ 


The objective is to find a policy $\pi ^*$ such that $V_{\pi^*}(x,y) \geq V_{\pi}(x,y)$ for any possible $\pi$ and any possible state $x,y$. {\color {red} {(proof for existence?)}}

\subsection{Modelling Approach}
For the sake of simplicity, we assume that the manager can only make decisions at fixed time epoch. That is consider the time horizon consists of $N$ small time periods $T=\{1,2,...N\}, N \leq \infty$. Between each decision epoch, the uncontrolled cash level will change by $\int^{t_2}_{t_1}(x_{ex}+x_{en})dt$ amount while the uncontrolled the asset level will not change. 

This cash management model can be formulated as a stationary Markov decision model with state space $X \times Y$ where $X$ and $Y$ are two sets of positive real numbers.  Between any two decision epoch $n$ and $n+1$, the uncontrolled cash-holding level $x_n$ will transit to $x_n'$ where


$$x_n' = x_n + \int_{t_n}^{t_{n+1}} x_{ex}(t) + x_{en}(t) dt= x_n + \theta + r y_n$$ 





with probability $$dp_\theta = d \mathbb{P} \left\{ \boldsymbol{N} = \theta \right\}  = \frac{1}{\sqrt{2\sigma^2\pi}} \exp \left\{ -\frac{(x-\mu)^2}{2\sigma^2}\right\} d\theta.$$ 


At each decision epoch, the manager could control the cash-holding level by moving any amount of money between the two accounts. And before the next decision epoch, if there exits cash shortage (i.e. a negative cash-holding level), the manager will be forced to sell his asset to offset the deficit as well as to pay the extra shortage penalty. To sum up, after one time period, the companies state will transit from $(x_n, y_n)$ to $(x_{n+1}, y_{n+1})$ such that:

if $x_n' + a_n - \Gamma_1 \cdot 1_{a_n < 0} \geq 0$
$$ \left\{ 
\begin{aligned}
& x_{n+1} = x_n' + a_n - \Gamma_1\cdot 1_{a_n < 0}\\
&y_{n+1} = y_n - a_n - \Gamma_2 \cdot 1_{a_n > 0}\\
\end{aligned}
\right.$$ 

 if $x_n' + a_n - \Gamma_1 \cdot 1_{a_n < 0} < 0$

$$ \left\{ 
\begin{aligned}
& x_{n+1} = 0\\
&y_{n+1} = y_n - a_n - \Gamma_2 \cdot 1_{a_n > 0} - SC(x_n', a_n) - x_n' - a_n + \Gamma_1\cdot 1_{a_n < 0} \\
\end{aligned}
\right.$$ with probability $p_\theta$.


We also consider a discount factor $\gamma$ for each time period. Value for future profit is normally discounted by the risk-free market interest. 
The objective of the model is to maximise the discounted net profit over the time horizon, i.e. to find a policy $\pi^*$ such that
$$a_i(x_0, y_0) = \arg\max_{a_i \in \pi^*} E\left\{ \sum_{i=1}^\infty \gamma^{i-1} \left(\theta + ry_{i-1} - \Gamma(a_i) - SC(x_i, a_i)  \right)\right\}$$
$$\forall x_0 \in X, y_0 \in Y$$

 For each time period, the immediate reward is given by $R_i^\pi (x_i, y_i)=  \theta + ry_{i-1} - \Gamma(a_i) - SC(x_i, a_i)$ and the Bellman equation can be written as: $$V_i^\pi(x_i, y_i) = R_i^\pi(x_i, y_i) + \gamma \int^\infty_{-\infty}p_\theta V_{i+1}^\pi(x_{i+1},y_{i+1})d \theta$$











\section{Insolvency Risk}
\section{Numerical Results}
\newpage
\bibliographystyle{plain}
\bibliography{BibliographyinCashProblem}
%\end{spacing}
\end{document}
